\documentclass[11pt,a4paper,]{article}
\usepackage{lmodern}

\usepackage{amssymb,amsmath}
\usepackage{ifxetex,ifluatex}
\usepackage{fixltx2e} % provides \textsubscript
\ifnum 0\ifxetex 1\fi\ifluatex 1\fi=0 % if pdftex
  \usepackage[T1]{fontenc}
  \usepackage[utf8]{inputenc}
\else % if luatex or xelatex
  \usepackage{unicode-math}
  \defaultfontfeatures{Ligatures=TeX,Scale=MatchLowercase}
\fi
% use upquote if available, for straight quotes in verbatim environments
\IfFileExists{upquote.sty}{\usepackage{upquote}}{}
% use microtype if available
\IfFileExists{microtype.sty}{%
\usepackage[]{microtype}
\UseMicrotypeSet[protrusion]{basicmath} % disable protrusion for tt fonts
}{}
\PassOptionsToPackage{hyphens}{url} % url is loaded by hyperref
\usepackage[unicode=true]{hyperref}
\hypersetup{
            pdftitle={Meat production and consumption in world},
            pdfborder={0 0 0},
            breaklinks=true}
\urlstyle{same}  % don't use monospace font for urls
\usepackage{geometry}
\geometry{a4paper, centering, text={16cm,24cm}}
\usepackage[style=authoryear-comp,]{biblatex}
\addbibresource{references.bib}
\usepackage{longtable,booktabs}
% Fix footnotes in tables (requires footnote package)
\IfFileExists{footnote.sty}{\usepackage{footnote}\makesavenoteenv{long table}}{}
\usepackage{graphicx,grffile}
\makeatletter
\def\maxwidth{\ifdim\Gin@nat@width>\linewidth\linewidth\else\Gin@nat@width\fi}
\def\maxheight{\ifdim\Gin@nat@height>\textheight\textheight\else\Gin@nat@height\fi}
\makeatother
% Scale images if necessary, so that they will not overflow the page
% margins by default, and it is still possible to overwrite the defaults
% using explicit options in \includegraphics[width, height, ...]{}
\setkeys{Gin}{width=\maxwidth,height=\maxheight,keepaspectratio}
\IfFileExists{parskip.sty}{%
\usepackage{parskip}
}{% else
\setlength{\parindent}{0pt}
\setlength{\parskip}{6pt plus 2pt minus 1pt}
}
\setlength{\emergencystretch}{3em}  % prevent overfull lines
\providecommand{\tightlist}{%
  \setlength{\itemsep}{0pt}\setlength{\parskip}{0pt}}
\setcounter{secnumdepth}{5}

% set default figure placement to htbp
\makeatletter
\def\fps@figure{htbp}
\makeatother

% NEW:set default table placement to htbp
\makeatletter
\def\fps@table{htbp}
\makeatother

\title{Meat production and consumption in world}

%% MONASH STUFF

%% CAPTIONS
\RequirePackage{caption}
\DeclareCaptionStyle{italic}[justification=centering]
 {labelfont={bf},textfont={it},labelsep=colon}
\captionsetup[figure]{style=italic,format=hang,singlelinecheck=true}
\captionsetup[table]{style=italic,format=hang,singlelinecheck=true}


%% FONT
\RequirePackage{bera}
\RequirePackage[charter,expert,sfscaled]{mathdesign}
\RequirePackage{fontawesome}

%% HEADERS AND FOOTERS
\RequirePackage{fancyhdr}
\pagestyle{fancy}
\rfoot{\Large\sffamily\raisebox{-0.1cm}{\textbf{\thepage}}}
\makeatletter
\lhead{\textsf{\expandafter{\@title}}}
\makeatother
\rhead{}
\cfoot{}
\setlength{\headheight}{15pt}
\renewcommand{\headrulewidth}{0.4pt}
\renewcommand{\footrulewidth}{0.4pt}
\fancypagestyle{plain}{%
\fancyhf{} % clear all header and footer fields
\fancyfoot[C]{\sffamily\thepage} % except the center
\renewcommand{\headrulewidth}{0pt}
\renewcommand{\footrulewidth}{0pt}}

%% MATHS
\RequirePackage{bm,amsmath}
\allowdisplaybreaks

%% GRAPHICS
\RequirePackage{graphicx}
\setcounter{topnumber}{2}
\setcounter{bottomnumber}{2}
\setcounter{totalnumber}{4}
\renewcommand{\topfraction}{0.85}
\renewcommand{\bottomfraction}{0.85}
\renewcommand{\textfraction}{0.15}
\renewcommand{\floatpagefraction}{0.8}


%\RequirePackage[section]{placeins}

%% SECTION TITLES


%% SECTION TITLES (NEW: Changing sections and subsections color)
\RequirePackage[compact,sf,bf]{titlesec}
\titleformat*{\section}{\Large\sf\bfseries\color[rgb]{0.8, 0.7, 0.1 }}
\titleformat*{\subsection}{\large\sf\bfseries\color[rgb]{0.8, 0.7, 0.1 }}
\titleformat*{\subsubsection}{\sf\bfseries\color[rgb]{0.8, 0.7, 0.1 }}
\titlespacing{\section}{0pt}{2ex}{.5ex}
\titlespacing{\subsection}{0pt}{1.5ex}{0ex}
\titlespacing{\subsubsection}{0pt}{.5ex}{0ex}


%% TITLE PAGE
\def\Date{\number\day}
\def\Month{\ifcase\month\or
 January\or February\or March\or April\or May\or June\or
 July\or August\or September\or October\or November\or December\fi}
\def\Year{\number\year}

%% LINE AND PAGE BREAKING
\sloppy
\clubpenalty = 10000
\widowpenalty = 10000
\brokenpenalty = 10000
\RequirePackage{microtype}

%% PARAGRAPH BREAKS
\setlength{\parskip}{1.4ex}
\setlength{\parindent}{0em}

%% HYPERLINKS
\RequirePackage{xcolor} % Needed for links
\definecolor{darkblue}{rgb}{0,0,.6}
\RequirePackage{url}

\makeatletter
\@ifpackageloaded{hyperref}{}{\RequirePackage{hyperref}}
\makeatother
\hypersetup{
     citecolor=0 0 0,
     breaklinks=true,
     bookmarksopen=true,
     bookmarksnumbered=true,
     linkcolor=darkblue,
     urlcolor=blue,
     citecolor=darkblue,
     colorlinks=true}

\usepackage[showonlyrefs]{mathtools}
\usepackage[no-weekday]{eukdate}

%% BIBLIOGRAPHY

\makeatletter
\@ifpackageloaded{biblatex}{}{\usepackage[style=authoryear-comp, backend=biber, natbib=true]{biblatex}}
\makeatother
\ExecuteBibliographyOptions{bibencoding=utf8,minnames=1,maxnames=3, maxbibnames=99,dashed=false,terseinits=true,giveninits=true,uniquename=false,uniquelist=false,doi=false, isbn=false,url=true,sortcites=false}

\DeclareFieldFormat{url}{\texttt{\url{#1}}}
\DeclareFieldFormat[article]{pages}{#1}
\DeclareFieldFormat[inproceedings]{pages}{\lowercase{pp.}#1}
\DeclareFieldFormat[incollection]{pages}{\lowercase{pp.}#1}
\DeclareFieldFormat[article]{volume}{\mkbibbold{#1}}
\DeclareFieldFormat[article]{number}{\mkbibparens{#1}}
\DeclareFieldFormat[article]{title}{\MakeCapital{#1}}
\DeclareFieldFormat[article]{url}{}
%\DeclareFieldFormat[book]{url}{}
%\DeclareFieldFormat[inbook]{url}{}
%\DeclareFieldFormat[incollection]{url}{}
%\DeclareFieldFormat[inproceedings]{url}{}
\DeclareFieldFormat[inproceedings]{title}{#1}
\DeclareFieldFormat{shorthandwidth}{#1}
%\DeclareFieldFormat{extrayear}{}
% No dot before number of articles
\usepackage{xpatch}
\xpatchbibmacro{volume+number+eid}{\setunit*{\adddot}}{}{}{}
% Remove In: for an article.
\renewbibmacro{in:}{%
  \ifentrytype{article}{}{%
  \printtext{\bibstring{in}\intitlepunct}}}

\AtEveryBibitem{\clearfield{month}}
\AtEveryCitekey{\clearfield{month}}

\makeatletter
\DeclareDelimFormat[cbx@textcite]{nameyeardelim}{\addspace}
\makeatother

\author{\sf\Large\textbf{ Di Cui}\\ {\sf\large Master of Business Analytics\\[0.5cm]} \sf\Large\textbf{ Guanru Chen}\\ {\sf\large Master of Business Analytics\\[0.5cm]} \sf\Large\textbf{ Yunzhi Chen}\\ {\sf\large Master of Business Analytics\\[0.5cm]}}

\date{\sf\Date~\Month~\Year}
\makeatletter
\lfoot{\sf Cui, Chen, Chen: \@date}
\makeatother


%%%% PAGE STYLE FOR FRONT PAGE OF REPORTS

\makeatletter
\def\organization#1{\gdef\@organization{#1}}
\def\telephone#1{\gdef\@telephone{#1}}
\def\email#1{\gdef\@email{#1}}
\makeatother
  \organization{Our World in Data}

  \def\name{TGIF \newline Di Cui \&\newline Guanru Chen \&\newline Yunzhi Chen}

  \telephone{(03) 9905 2478}

  \email{questions@company.com}                 %NEW: New email addresss

\def\webaddress{\url{http://company.com/stats/consulting/}} %NEW: URl
\def\abn{12 377 614 630}                                    % NEW: ABN
\def\logo{\includegraphics[width=6cm]{Figures/WeChat Image_20220521114617}}  %NEW: Changing logo
\def\extraspace{\vspace*{1.6cm}}
\makeatletter
\def\contactdetails{\faicon{phone} & \@telephone \\
                    \faicon{envelope} & \@email}
\makeatother

%%%% FRONT PAGE OF REPORTS

\def\reporttype{Report for}

\long\def\front#1#2#3{
\newpage
\begin{singlespacing}
\thispagestyle{empty}
\vspace*{-1.4cm}
\hspace*{-1.4cm}
\hbox to 16cm{
  \hbox to 6.5cm{\vbox to 14cm{\vbox to 25cm{
    \logo
    \vfill
    \parbox{6.3cm}{\raggedright
      \sf\color[rgb]{0.8, 0.7, 0.1 }    % NEW color 
      {\large\textbf{\name}}\par
      \vspace{.7cm}
      \tabcolsep=0.12cm\sf\small
      \begin{tabular}{@{}ll@{}}\contactdetails
      \end{tabular}
      \vspace*{0.3cm}\par
      ABN: \abn\par
    }
  }\vss}\hss}
  \hspace*{0.2cm}
  \hbox to 1cm{\vbox to 14cm{\rule{4pt}{26.8cm}\vss}\hss\hfill}  %NEW: Thicker line
  \hbox to 10cm{\vbox to 14cm{\vbox to 25cm{   
      \vspace*{3cm}\sf\raggedright
      \parbox{11cm}{\sf\raggedright\baselineskip=1.2cm
         \fontsize{24.88}{30}\color[rgb]{0, 0.29, 0.55}\sf\textbf{#1}}   % NEW: title color blue
      \par
      \vfill
      \large
      \vbox{\parskip=0.8cm #2}\par
      \vspace*{2cm}\par
      \reporttype\\[0.3cm]
      \hbox{#3}%\\[2cm]\
      \vspace*{1cm}
      {\large\sf\textbf{\Date~\Month~\Year}}
   }\vss}
  }}
\end{singlespacing}
\newpage
}

\makeatletter
\def\titlepage{\front{\expandafter{\@title}}{\@author}{\@organization}}
\makeatother

\usepackage{setspace}
\setstretch{1.5}

%% Any special functions or other packages can be loaded here.
\usepackage{booktabs}
\usepackage{longtable}
\usepackage{array}
\usepackage{multirow}
\usepackage{wrapfig}
\usepackage{float}
\usepackage{colortbl}
\usepackage{pdflscape}
\usepackage{tabu}
\usepackage{threeparttable}
\usepackage{threeparttablex}
\usepackage[normalem]{ulem}
\usepackage{makecell}
\usepackage{xcolor}


\begin{document}
\titlepage

\hypertarget{data-set-introduction}{%
\section{Data set introduction}\label{data-set-introduction}}

Meat is an important source of nutrition for many people around the world. Meat production and consumption greatly affect the sustainable development of the world.

With the continuous development of the global economy and the continuous improvement of people's living standards, the market has numerous demands for meat products, with more and more variety requirements and higher quality requirements. So that we could better understand the situation about the global meat production.

The data set was obtained from the ``Our World in Data'' data base and contains the following variables regarding to meat production of the world:

\begin{itemize}
\tightlist
\item
  Entity
\item
  Year
\item
  Amount of production: Measure in tons.
\item
  Livestock types: including beef and buffalo, pigmeat, poultry, sheep and goat, and other meat types such as horse or camel et cetera.
\item
  Meat type consumption (kg/capita/year).
\end{itemize}

\hypertarget{research-questions}{%
\section{Research questions}\label{research-questions}}

\begin{itemize}
\tightlist
\item
  How did global meat production develop by continents?
\item
  How did the meat production develop in some countries which contributed greatly?
\item
  What's the production distribution of different livestock types across the world?
\item
  Which countries are main production country for different types of meat, such as: Beef and buffalo, pig and poultry?
\item
  Which countries eat the most meat in the last 20 years?
\item
  What types of meat do people eat?
\end{itemize}

\clearpage

\hypertarget{exploratory-data-analysis}{%
\section{Exploratory data analysis}\label{exploratory-data-analysis}}

\section*{Global meat production -- Di Cui}

\subsection*{Analysis}

\textbf{Q1: How did global meat production develop by continents?}

\begin{figure}
\centering
\includegraphics{report_files/figure-latex/continent-figure-1.pdf}
\caption{\label{fig:continent-figure}Global meat production from 1961 to 2018}
\end{figure}

From Figure\ref{fig:continent-figure}, we can see all continents show an uptrend. In particular, production in Asia has increased from around 10 million tonnes to around 150 million tonnes. And in 1992, Asia produced more than Europe and became the largest meat producer.

\clearpage

\begin{figure}
\centering
\includegraphics{report_files/figure-latex/continent-increase-1.pdf}
\caption{\label{fig:continent-increase}Comparision of meat production between 1961 and 2018}
\end{figure}

See Figure\ref{fig:continent-increase}. In 2018, meat production in Asia has increased around 15 times, accounting for around 43\% of total production, while the proportion of Europe and Northern America has decreased, although their production has increased.

\clearpage

\textbf{Q2: How did the meat production develop in some countries which contributed greatly?}

\begin{figure}
\centering
\includegraphics{report_files/figure-latex/country-1.pdf}
\caption{\label{fig:country}Top six countries' meat production from 1961 to 2018}
\end{figure}

From Figure\ref{fig:country}, it is obvious that China meat production has increased sharply. And China surpassed the United States in 1990, and has been the largest meat production country in the world since 1990.

\clearpage

\begin{figure}
\centering
\includegraphics{report_files/figure-latex/country-increase-1.pdf}
\caption{\label{fig:country-increase}Comparision of meat production between 1961 and 2018}
\end{figure}

See Figure\ref{fig:country-increase}. In 2018, The world's meat production has increased around quadrupled. And China has increased about 34 times compared to 1961, accounting for about 26\% of global meat production. And the meat production in United States accounted for about 14\%, although the production only has doubled.

\clearpage

\section*{Meat production by livestock type-- Guan Ru Chen}

\subsection*{Research Question}

\begin{enumerate}
\def\labelenumi{\arabic{enumi}.}
\item
  What's the production distribution of different livestock types across the world?
\item
  Which countries are main production country for different types of meat, such as: Beef and buffalo, pig and poultry?
\end{enumerate}

\subsection*{Analysis}

\textbf{Q1: What's the production distribution of different livestock types across the world?}

\begin{figure}
\centering
\includegraphics{report_files/figure-latex/productiontype-1.pdf}
\caption{\label{fig:productiontype}Global meat production by livestock type, 1961 to 2018}
\end{figure}

In figure \ref{fig:productiontype}, we see that the dominant livestock types are poultry, cattle (which includes beef and buffalo meat), pig, and sheep \& goat to a lesser extent at global level.

Although production of all major meat types have been increasing in absolute terms, in relative terms the share of global meat types have changed significantly over the last 50 years. In 1961, poultry meat accounted for small portion; by 2013 its share has tripled. In comparison, beef and buffalo meat as a share of total meat production has nearly halved. And the Pigmeat's share has remained more constant.

\clearpage

\textbf{Q2: Which countries are main production country for different types of meat, such as: Beef and buffalo, pig and poultry? }

\begin{table}

\caption{\label{tab:cattle}Beef and buffalo (cattle) meat production (million tonnes), 1961-2018}
\centering
\begin{tabular}[t]{lr}
\toprule
Entity & Production\\
\midrule
World & 3048.30\\
Americas & 1355.77\\
Europe & 808.42\\
Northern America & 687.74\\
United States & 627.73\\
\addlinespace
South America & 568.55\\
Asia & 527.12\\
European Union & 494.53\\
Eastern Europe & 340.15\\
Low Income Food Deficit Countries & 296.90\\
\addlinespace
Net Food Importing Developing Countries & 291.25\\
Brazil & 289.01\\
Europe, Western & 243.83\\
Africa & 227.54\\
Southern Asia & 195.48\\
\addlinespace
USSR & 192.97\\
Eastern Asia & 192.05\\
Argentina & 155.12\\
China & 151.85\\
Least Developed Countries & 140.57\\
\bottomrule
\end{tabular}
\end{table}

\includegraphics{report_files/figure-latex/cattle-1.pdf}

\begin{table}

\caption{\label{tab:poultry}Poultry meat production (million tonnes), 1961-2018}
\centering
\begin{tabular}[t]{lr}
\toprule
Entity & Production\\
\midrule
World & 2971.53\\
Americas & 1247.97\\
Asia & 925.59\\
Northern America & 714.31\\
United States & 669.00\\
\addlinespace
Europe & 628.97\\
Eastern Asia & 514.42\\
European Union & 476.77\\
China & 428.55\\
South America & 420.22\\
\addlinespace
Brazil & 264.62\\
Eastern Europe & 222.83\\
Net Food Importing Developing Countries & 207.48\\
South Eastern Asia & 191.20\\
Europe, Western & 183.20\\
\addlinespace
Southern Europe & 135.95\\
Africa & 134.62\\
Southern Asia & 121.39\\
Low Income Food Deficit Countries & 119.84\\
Western Asia & 95.01\\
\bottomrule
\end{tabular}
\end{table}

\includegraphics{report_files/figure-latex/poultry-1.pdf}

\begin{table}

\caption{\label{tab:pig}Pig meat production (million tonnes), 1961-2018}
\centering
\begin{tabular}[t]{lr}
\toprule
Entity & Production\\
\midrule
World & 4097.38\\
Asia & 1898.60\\
Eastern Asia & 1669.83\\
China & 1560.96\\
Europe & 1381.51\\
\addlinespace
European Union & 1058.78\\
Americas & 757.20\\
Northern America & 523.61\\
Europe, Western & 506.08\\
Eastern Europe & 469.05\\
\addlinespace
United States & 450.61\\
Germany & 243.52\\
Southern Europe & 237.61\\
South Eastern Asia & 201.43\\
Northern Europe & 168.77\\
\addlinespace
South America & 161.54\\
USSR & 158.90\\
Low Income Food Deficit Countries & 122.94\\
Spain & 112.06\\
France & 107.51\\
\bottomrule
\end{tabular}
\end{table}

\includegraphics{report_files/figure-latex/pig-1.pdf}

In the table \ref{tab:cattle}, we see the global production of cattle (beef and buffalo) meat. From the country's perspective, The United States is the world's largest beef and buffalo meat producer. Other major producers are Brazil and China.

In the table \ref{tab:poultry}, we can see the production of poultry, like cattle production, the United States is still the world's largest producer. China and Brazil are also large poultry producers. Collectively, Europe is also a major poultry producer, just below the United States.

But for pigmeat production \ref{tab:pig}, China dominates global output, producing just short of half of total pigmeat. The other major producers include the United States, Germany.
\clearpage

\section*{Per capita meat consumption -- Yunzhi Chen}

\subsection*{Analysis}

\textbf{Q1: Which countries eat the most meat in the last 20 years? }

As can be seen from the Table \ref{tab:highest-meanconsumption} below, the top six countries with the highest average meat consumption mean in the world over the 20-year period from 1997 to 2017 are the United States, Australia, New Zealand, Spain, French Polynesia and Bahamas. The highest per capita meat consumption mean of the United States reached about 121.2 kg/capita/year. It can be concluded that countries with high income also consume more meat. Developed countries account for a large share of the six countries with the highest average meat consumption over years.

\begin{table}

\caption{\label{tab:highest-meanconsumption}Top 6 countries with the largest mean of meat consumption over years}
\centering
\begin{tabular}[t]{l|r}
\hline
Country & Mean\_consumption\_kg\_capita\_yr\\
\hline
United States & 121.19\\
\hline
Australia & 114.96\\
\hline
New Zealand & 105.19\\
\hline
Spain & 105.14\\
\hline
French Polynesia & 99.86\\
\hline
Bahamas & 98.57\\
\hline
\end{tabular}
\end{table}
\clearpage

In terms of changing trends, figure \ref{fig:highest-consumption-trend} shows the meat consumption per person in these countries over the last 20 years have fluctuated considerably, with the exception of Australia and the United States, where consumption has increased, while all other countries have shown a decreasing trend, but the total value is still much higher than the world per capita meat consumption.

\begin{figure}
\centering
\includegraphics{report_files/figure-latex/highest-consumption-trend-1.pdf}
\caption{\label{fig:highest-consumption-trend}Consumption trend of top 6 countries}
\end{figure}

\clearpage

\textbf{Q2: What types of meat do people eat?}

From figure \ref{fig:Meat-type}, it illustrates that as a global average, pork has the highest per capita consumption of meat commodities; in 2013, per capita pork consumption was about 16 kg; followed by 15 kg of poultry; 9 kg of beef/buffalo meat; 2 kg of lamb and goat; and only a small percentage of other meats.

\begin{figure}
\centering
\includegraphics{report_files/figure-latex/Meat-type-1.pdf}
\caption{\label{fig:Meat-type}Meat type changes over time}
\end{figure}

\clearpage

\hypertarget{conclusion}{%
\section{Conclusion}\label{conclusion}}

\begin{itemize}
\item
  In 2018, global meat production has increased around quadrupled, and Asia and China have contributed greatly to global meat production.
\item
  The average person in the world consumed around 43 kilograms of meat in 2014. This ranges from over 100kg in the US and Australia to only 5kg in India.
\item
  The amount of meat produced for a given animal varies significantly across the world based on production systems.
\item
  Richer countries tend to consume more meat per person. Developed countries account for a large share of the six countries with the highest average meat consumption in the last 20 years.
\item
  Although there is a lot of fluctuation, the amount of meat consumed per capita is much larger than the world average.
\item
  At the world level, per capita consumption of pork is the highest among meat commodities over years.
\end{itemize}

\section*{Citation}

The data set is cited from \textcite{owidmeatproduction}.

Analysis of the data is done using the following packages:

bookdown \textcite{bookdown1}, \textcite{bookdown2},

tidyverse \textcite{tidyverse},

readr \textcite{readr},

viridis \textcite{viridis},

gridExtra \textcite{gridExtra}

\printbibliography

\end{document}
